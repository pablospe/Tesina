\section{Feature extraction}
\label{feature_extraction}

Las t�cnicas usuales para \textit{handwriting recognition} tratan de encontrar \textit{features} particulares sobre un conjunto de s�mbolos, citemos por ejemplo los d�gitos. Pero al cambiar dicho conjunto, estos features dejan de ser efectivos, no discriminan correctamente. Entonces, se vuelve impr�ctico desarrollar heur�sticas para reconocer features espec�ficos para cada s�mbolo.
% Sobre todo si se tiene en cuenta que s�mbolos matem�ticos pueden ser inventados o agregados en la marcha.
Por lo que es deseable buscar una representaci�n que permita ser aplicada sin importar el s�mbolo que se trate; ya sea un d�gito, una letra o un s�mbolo matem�tico.

Unos de los mayores problemas con los m�todos de reconocimientos tradicionales es que los trazos son tratados como secuencias de puntos (disceto), en vez de verlos como lo que realmente son, curvas (continuo).
% son pensados como una secuencia de puntos. El problema con �sto es que no se lo est�n tratando como lo que realmente son, curvas.