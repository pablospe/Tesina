                        %%%%%   ESTILO  %%%%%

\renewcommand{\headrulewidth}{0pt}  %  l�nea horiz. bajo el encabezado
\renewcommand{\footrulewidth}{0pt}  %  l�nea horiz. sobre el pie

\pagenumbering{arabic}

%
%  Simple faz
%
\pagestyle{fancy}
\fancyhf{}                   % Borrar todos los ajustes
\fancyfoot[C]{\thepage}
% \fancyhead[RO,RE]{\scshape{\thepage}}  % N�meros de p�gina en las esquinas de los encabezados
\renewcommand{\sectionmark}[1]{
    \fancyhead[RO]{\begin{footnotesize}\thesection.\ \scshape{#1}\end{footnotesize}}
    \fancyhead[RE]{\begin{footnotesize}\thesection.\ \scshape{#1}\end{footnotesize}}
    \renewcommand{\headrulewidth}{0.1pt}
}
\renewcommand{\subsectionmark}[1]{
    \fancyhead[RE]{\begin{footnotesize}\thesubsection.\ \scshape{#1}\end{footnotesize}}
}


%
%  Doble faz
%
% \pagestyle{fancy}
% \fancyhf{}                   % Borrar todos los ajustes
% \fancyfoot[C]{\thepage}
% % \fancyhead[RO,LE]{\begin{small}\thepage\end{small}}  % N�meros de p�gina en las esquinas de los encabezados
% \renewcommand{\sectionmark}[1]{
%     \fancyhead[RO]{\begin{footnotesize}\thesection.\ \scshape{#1}\end{footnotesize}}
%     \fancyhead[LE]{\begin{footnotesize}\thesection.\ \scshape{#1}\end{footnotesize}}
%     \renewcommand{\headrulewidth}{0.1pt}
% }
% \renewcommand{\subsectionmark}[1]{
%     \fancyhead[RE]{\begin{footnotesize}\thesubsection.\ \scshape{#1}                   \end{footnotesize}}
% }


%% Para los perzonalizar itemize
\renewcommand{\labelitemi}{$\bullet$}
\renewcommand{\labelitemii}{$\ast$}


% % A more convenient way to stop floats at section boundaries is to change
% % the definition of "\section" to include "\FloatBarrier", at the beginning
% \let\oldsection\section
% \renewcommand{\section}{\FloatBarrier\oldsection}


% % Util para ser impreso. Hace que las secciones comiencen desde una p�gina impar
% % similar a \cleardoublepage
% \let\oldsection\section
% \renewcommand{\section}{\cleartoevenpage\oldsection}

