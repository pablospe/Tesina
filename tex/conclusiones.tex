\section{Conclusiones}
\label{sec:conclusion}

La principal motivaci�n de este trabajo fue la intenci�n de reconocer \textbf{f�rmulas matem�ticas}. Dado que el campo es complejo, el trabajo debi� focalizarse en la identificaci�n individual de s�mbolos, paso previo y necesario para el objetivo final de reconocimiento de expresiones matem�ticas. Un requerimiento importante de los features es que sean \textbf{robustos}, en el sentido de poder usarse ya sea para d�gitos, letras o s�mbolos matem�ticos cualesquiera\footnote{Cabe destacar que posiblemente se tenga que tratar los signos de puntuaci�n de manera discriminada por su extremada corta longitud.}. Lo cual se consigui� gracias a la representaci�n utilizada.

Una observaci�n importante es que en este trabajo solo se consider� el \textbf{primer match}, lo cual no es realista en el escenario de f�rmulas matem�ticas pues muchos s�mbolos quedan totalmente definidos dentro de un contexto. Como ejemplo, podr�a pensarse en una $S$ (algo m�s estirada) y en una $\int$ (integral), donde la desambiguaci�n se produce cuando uno escribe los extremos de la integral o un diferencial, no quedando dudas de qu� s�mbolo se trate.

Se logr� en este trabajo una implementaci�n muy eficiente del c�lculo de los momentos, gracias a pensar los algoritmos como online, sin esperar la finalizaci�n de una etapa para comenzar la siguiente, aprovechando as� los tiempos muertos que el usuario deja mientras escribe, se\-cci�n~\ref{sec:calculo_numerico_momentos}.

Tambi�n se ha mostrado que una representaci�n con polinomios ortogonales caracteriza muy bien a los trazos, permitiendo alcanzar una alta precisi�n en el reconocimiento. Los resultados conseguidos con los polinomios de \textbf{Legendre-Solobev} fueron sobresalientes, acerc�ndose al �ptimo ($99.5\%$) en la base de datos de d�gitos, secci�n \ref{sec:mejor_representacion}.

En cuanto a la clasificaci�n, se ha mostrado que \textbf{SVM} alcanza los mejores resultados; avalando por qu� este m�todo, gracias a su desarrollo te�rico subyacente, es de los m�s usados actualmente en gran variedad de problemas.

En resumen, se han utilizado m�todos modernos para la representaci�n de los trazos, diferenci�ndose de los m�todos tradicionales de reconocimiento de escritura online en los cuales los trazos son tratados como secuencias de puntos, en lugar de curvas continuas. Esta representaci�n permite obtener excelentes resultados tanto en la \textbf{precisi�n} de reconocimiento como as� tambi�n en la \textbf{eficiencia} computacional.

