%% Entorno Descripci�n. En tres sabores:
    %% 1. El m�s general
    \newenvironment{descripcion}[2]%
    {\begin{basedescript}{\desclabelwidth{#2}}%
    \item[#1]}%
    {\end{basedescript}}%

    %% 2. Como poner en negrita (tambi�n parecido a section* pero puedo seguir en la misma l�nea)
    \newenvironment{descripcion0}[1]%
    {\begin{basedescript}{\desclabelwidth{-1ex}}%
    \item[#1]}%
    {\end{basedescript}}%

    %% 3. Con 1.25cm de tama�o
    \newenvironment{descripcion1}[1]%
    {\begin{basedescript}{\desclabelwidth{1.25cm}}%
    \item[#1]}%
    {\end{basedescript}}%

    %% Entorno para citar texto, 80% del tama�o (10% de cada lado)
      \newenvironment{citetext}{%
      \def\leftmargini{0.1\textwidth}%
      \def\rightmargini{0.1\textwidth}%
      \vspace*{-0.35cm}%
      \begin{quotation}\parskip 0.15cm\guillemotleft}%
      {\guillemotright\end{quotation}\vspace*{0.1cm}}


      %% Sin � (caso especial de citetext)
      \newenvironment{citetext_sin_right}{%
      \def\leftmargini{0.1\textwidth}%
      \def\rightmargini{0.1\textwidth}%
      \vspace*{-0.35cm}%
      \begin{quotation}\parskip 0.15cm\guillemotleft}%
      { \end{quotation}\vspace*{0.1cm}}



\newcommand{\subsubseccion}[1]{\subsubsection*{#1}\addcontentsline{toc}{subsubsection}{#1}}




\newcommand{\cleartoevenpage}{%   Similar a \cleardoublepage
    \clearpage
    \ifodd\thepage
        \newpage
    \else
        \newpage
        ~ \\
        \newpage
    \fi
}




\newcommand{\refEQ}[1]{{\color{red} (\ref{#1})}}

