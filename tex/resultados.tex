\section{Resultados}

\subsection{Bases de datos disponibles}

\subsection{Momentos vs Pseudo-inversa}
\subsection{Preprocesar o No Preprocesar}
\label{prepoceso}

Como se indic� en la secci�n \ref{sec:conceptos_preproceso}, antes del c�lculo de los features se suelen usar~$3$ filtros como preproceso. Estos filtros son: 1. suavizado, 2. resizing, y 3. resampling (no necesariamente en ese orden). En lo siguiente se desea examinar la posibilidad de eliminar estos pasos en pos de eficiencia. Evitar utilizar estos filtros permitir�a calcular los momentos a medida que se van ingresando los datos, como se ha indicado en la secci�n \ref{sec:calculo_numerico_momentos}, lo cu�l permitir�a un aumento considerable de eficiencia en comparaci�n con el m�todo de la pseudo-inversa.

\subsection{Evitando suavizado}
Recordar que los features son los coeficientes de polinomios, obtenidos por aproximaci�n de m�nimos cuadrados, lo cual genera una curva suave. El objetivo del suavizado (la eliminaci�n del ruido presente al principio y final de cada trazo) es alcanzado sin necesidad de realizar un filtrado.

\subsection{Evitando resizing}
Se ha demostrado que los features obtenidos son invariantes a escala, secci�n \ref{Invariante_escala}. Por lo que no es innecesario aplicar este filtro.






\subsection{Parametrizaci�n por tiempo vs Parametrizaci�n por longitud de arco}
\subsection{Representaci�n a elegir: Legendre, Chebyshev o Legendre-Sobolev}

\subsection{Momentos como features}

\subsection{Mejor performance}
\subsection{Mejor precisi�n}

% \begin{figure}[h!]
%  \centering
%  \advance\leftskip-2.1cm
%  \advance\rightskip-2.1cm
%  \includegraphics[scale=0.42,keepaspectratio=true]{/home/spe/projects/legendre/least_square_L_without_prepocess.pdf}
% % \end{figure}
% % \begin{figure}[h!]
% %  \centering
% \hspace*{-0.4cm}
% \includegraphics[scale=0.41,keepaspectratio=true]{/home/spe/projects/legendre/least_square_L.pdf}
% \end{figure}