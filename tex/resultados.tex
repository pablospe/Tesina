\section{Resultados}

\subsection{Bases de datos disponibles}

\subsection{Momentos vs Pseudo-inversa}
\subsection{Preprocesar o No Preprocesar}
\label{prepoceso}

Como se indic� en la secci�n \ref{sec:conceptos_preproceso}, antes del c�lculo de los features se suelen usar~$3$ filtros como preproceso. Estos filtros son: 1. suavizado, 2. resampling, y 3. resizing (no necesariamente en ese orden). En lo siguiente se desea examinar la posibilidad de eliminar estos pasos en pos de eficiencia.

%Evitar utilizar estos filtros permitir�a el c�lculo de los momentos a medida que se van ingresando los datos, como se ha indicado en la secci�n \ref{sec:calculo_numerico_momentos}; lo cu�l permitir�a un aumento considerable de eficiencia en comparaci�n con el m�todo de la pseudo-inversa, el cual necesita esperar a que el usuario termine


\subsubsection{Evitando suavizado}
Recordar que los features son los coeficientes de polinomios, obtenidos por aproximaci�n de m�nimos cuadrados, lo cual genera una curva suave. El objetivo del suavizado (la eliminaci�n del ruido presente al principio y final de cada trazo) es alcanzado sin necesidad de realizar un filtrado. Entonces, no es necesario este paso.

\subsubsection{Evitando resampling}
Al cambiar la representaci�n de los trazos, de secuencia de puntos a curvas continuas, no hay necesidad de espaciarlos uniformemente o reducir/aumentar la cantidad de puntos en los trazos (tarea del filtro en cuesti�n). Si la velocidad en la que es escrito un s�mbolo afecta el reconocimiento, se puede optar por la reparametrizaci�n por longitud de arco, secci�n \ref{sec:arc-length}. Por lo tanto, tampoco es necesario el resampling.

\subsubsection{Evitando resizing}
Se ha demostrado que los features obtenidos son invariantes a escala, secci�n \ref{Invariante_escala}. Por lo que tambi�n es innecesario aplicar este filtro.

\subsubsection{Evitando traslaci�n}
Un filtro no mencionado es el de traslaci�n, que se ocupa de trasladar el trazo al origen; �sto es, que los m�nimos en cada eje sea $0$. Se ha expresado que los features no son invariantes a traslaci�n. Aqu� se muestra por experimentaci�n, que las traslaciones no provocan una p�rdida significativa en la precisi�n de reconocimiento. �sto permitir�a tomar dos caminos al implementar un sistema de reconocimiento: preprocesar o no hacerlo. Esta decisi�n de dise�o puede ser tomada para caso en que se requiera extrema eficiencia, como en el caso de los dispositivos m�viles, a costa de perder precisi�n. Entonces, se tiene que evaluar en qu� escenario se desea utilizar el sistema de reconocimiento, y elegir el balance justo entre eficiencia y precisi�n seg�n sea el caso.



\begin{figure}[h!]
 \centering
 \advance\leftskip-2.8cm
 \advance\rightskip-2.8cm
 \includegraphics[scale=0.458,keepaspectratio=true]{imagen/plot/least_square_L_preproceso_0.pdf}
% \end{figure}
% \begin{figure}[h!]
%  \centering
\hspace*{-0.4cm}
\end{figure}





\subsection{Parametrizaci�n por tiempo vs Parametrizaci�n por longitud de arco}
\subsection{Representaci�n a elegir: Legendre, Chebyshev o Legendre-Sobolev}

\subsection{Momentos como features}

\subsection{Mejor performance}
\subsection{Mejor precisi�n}
