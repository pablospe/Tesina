\begin{center}
\begin{LARGE}\textbf{\textsc{Resumen}}\end{LARGE}
\end{center}

% \noindent
% En este trabajo s
% Se utilizaron m�todos de reconocimiento de escritura que permitieron alcanzar una alta efectividad de reconocimiento en las bases de datos de s�mbolos usadas. Puff...

% otro pufff....
\noindent
Se intenta reconocer escritura obtenida digitalmente como secuencias de puntos de una manera robusta; �sto es, reconocer trazos indistintamente de que sean d�gitos, letras o s�mbolos matem�ticos. Se probaron diferentes m�todos basados en la idea de tratar a las secuencias de puntos como curvas continuas. Lo cual es posible una vez que se aproximan los trazos con bases de polinomios ortogonales. Dichas aproximaciones caracterizan a los trazos, permitiendo obtener una representaci�n que es usada en la etapa de clasificaci�n y reconocimiento.
% lo suficientemente bien como para permitir obtener



\newpage