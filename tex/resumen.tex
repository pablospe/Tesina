\begin{center}
\begin{LARGE}\textbf{Resumen}\end{LARGE}
\end{center}

\noindent
Este trabajo tiene como objetivo reconocer escritura manuscrita obtenida digitalmente como secuencias de puntos de una manera robusta; �sto es, reconocer trazos indistintamente de que sean d�gitos, letras o s�mbolos matem�ticos. Se probaron diferentes m�todos basados en la misma idea: tratar las secuencias de puntos como curvas continuas, lo cual es posible aproximando los trazos mediante bases de polinomios ortogonales. Se probar� que dichas aproximaciones caracterizan muy bien a los trazos, permitiendo alcanzar una alta precisi�n en el reconocimiento, y eficiencia computacional. Se obtuvieron buenos resultados en dos bases de datos diferentes, una de d�gitos y otra de letras.

\newpage